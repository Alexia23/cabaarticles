
%%% Local Variables:
%%% mode: latex
%%% TeX-master: t
%%% End:

\chapter{主要结论和进一步研究工作}
\section{主要结论}
\begin{itemize}
\item 在全部部署的情况下,FIB表的表项数目是现网络环境下FIB表表项的1\/10,增量部署的情况下,FIB表表项的压缩情况与部署自治系统向外宣布的前缀数目相关,部署自治系统向外宣告的前缀数目越多,FIB表表项的压缩情况越好。
\item 一个自治系统平均向外宣布10条前缀,部署CABA地址分配方案的自治系统,向外宣布嵌有ASN的直到网络路由收敛的UPDATE 的数目平均是没有部署向外宣布现今网络中的前缀直到网络路由收敛的UPDATE数目的$1/10$。部署CABA方案的自治系统只需要向外宣布一条前缀,与现今网络中一个自治系统需要向外宣告n条前缀相比,update减少$(n-1)/n$。
\item 基于CABA编址BGP路由表项的数目将会远小于现有对应的BGP路由表项。
\end{itemize}
\section{进一步研究工作}
\subsection{根据网络层级设计增量部署计算FIB压缩率}
在第\ref{cha:compression}章的实验中,可以根据网络层级进行部署,计算FIB的压缩率。
\subsection{SIMBGP仿真tier1自治系统观察收敛时间}
在第\ref{cha:simu}章的SIMBGP仿真实验中,可以由Tier的自治系统向外宣告路由,观察其UPDATE数目及其收敛时间,也就是在仿真选取自治系统时,可以根据层级结构进行随机选取,而不是完全随机。
\subsection{Docker部署更大的网络规模}
目前Docker上只部署了17台路由器,为了接近网络的真实环境,可以部署几百台甚至上千台进行实验。
\subsection{部署实际网络环境,分析实验}
本文所有的实验均是虚拟的网络环境,可能和真实的网络环境仍有出入,所以应该部署实际的网络环境,进行实验的分析。
