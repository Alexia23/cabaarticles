
%%% Local Variables:
%%% mode: latex
%%% TeX-master: t
%%% End:
\secretlevel{绝密} \secretyear{2100}

\ctitle{基于AS编址的互联网可扩展路由机制的性能评价}
% 根据自己的情况选,不用这样复杂
\makeatletter
\ifthu@bachelor\relax\else
  \ifthu@doctor
    \cdegree{工学博士}
  \else
    \ifthu@master
      \cdegree{工学硕士}
    \fi
  \fi
\fi
\makeatother


\cdepartment[计算机]{计算机科学与技术系}
\cmajor{计算机科学与技术}
\cauthor{王庆}
\csupervisor{王之梁副教授}
% 如果没有副指导老师或者联合指导老师,把下面两行相应的删除即可。
%\cassosupervisor{施新刚副教授}
% \ccosupervisor{某某某教授}
% 日期自动生成,如果你要自己写就改这个cdate
%\cdate{\CJKdigits{\the\year}年\CJKnumber{\the\month}月}

% 博士后部分
% \cfirstdiscipline{计算机科学与技术}
% \cseconddiscipline{系统结构}
% \postdoctordate{2009年7月——2011年7月}

\etitle{Bandwidth Guarantee for Tenant Networks in Datacenter Clouds}
% 这块比较复杂,需要分情况讨论:
% 1. 学术型硕士
%    \edegree:必须为Master of Arts或Master of Science(注意大小写)
%              “哲学、文学、历史学、法学、教育学、艺术学门类,公共管理学科
%               填写Master of Arts,其它填写Master of Science”
%    \emajor:“获得一级学科授权的学科填写一级学科名称,其它填写二级学科名称”
% 2. 专业型硕士
%    \edegree:“填写专业学位英文名称全称”
%    \emajor:“工程硕士填写工程领域,其它专业学位不填写此项”
% 3. 学术型博士
%    \edegree:Doctor of Philosophy(注意大小写)
%    \emajor:“获得一级学科授权的学科填写一级学科名称,其它填写二级学科名称”
% 4. 专业型博士
%    \edegree:“填写专业学位英文名称全称”
%    \emajor:不填写此项
\edegree{Master of Science}
\emajor{Computer Science and Technology}
\eauthor{Wang Jingyi}
\esupervisor{Professor Yin Xia}
%\eassosupervisor{Shi Xingang}
% 这个日期也会自动生成,你要改么?
% \edate{December, 2005}

% 定义中英文摘要和关键字
\begin{cabstract}
  互联网的不断完善和扩展,网络设备和终端设备数量的不断增加,IPv4地址处于枯竭的状态,从IPv4过渡到IPv6的技术逐渐成熟,IPv6 地址本身的诸多优势,都预示未来IPv6网络将会迅速发展,由此带来的IPv6网络路由的可扩展性、收敛性和安全性问题值得研究。

  本文为了解决IPv6网络带来的可扩展路由问题,提出了基于无类AS编址的CABA(Classless AS Based Addressing)方案,独立于现有网络的IPv6地址分配方案,同时能够兼容现有的IPv6方案,可以实现增量部署。新型的CABA方案,将自治系统号嵌入到IPv6的地址中,在BGP路由的过程中,一个自治系统只需要向外广播一条嵌入自治系统号的前缀,这将极大程度的减少全局路由表的大小,同时也会减少BGP中update包的数目以及路由震荡的频率和收敛时间,尽可能解决IPv6网络带来的可扩展路由问题。

  本文主要进行了以下工作;

  \begin{enumerate}
  \item 提出了一个新型的基于无类AS的IPv6编址方案。同时为了实现IPv6网络域间路由的扩展性,设计了基于自治系统号的域间路由机制A-BGP(ASN based BGP)。
  \item 将当前网络中的RIB表转换成FIB,与A-BGP得到的FIB表进行对比,评估CABA和A-BGP下FIB 表的压缩情况,结果显示A-BGP下的FIB 表项是现网络中FIB表的10%。
  \item 熟悉SIMBGP仿真系统,加载CAIDA上全网的拓扑结构,任选多个自治系统,对比其向外广播嵌入自治系统号的IPv6前缀和广播现在网络中广播出去的所有IPv4 的update包数和收敛时间,结果显示当前网络自治系统号广播出去的prefix越多,采用新型A-BGP机制update数目减少幅度越大。
  \item 通过Docker软件运行多台装有Quagga软件路由器的虚拟机器,部署小型拓扑结构,在该结构上实现A-BGP机制,通过查看生成的BGP路由表表项数目,评估路由的可扩展程度。
  \end{enumerate}
\end{cabstract}

\ckeywords{CABA, A-BGP, 可扩展路由}

\begin{eabstract}
Waiting for translation.

The main work of this thesis include:
\begin{enumerate}
\item Propose a new design.
\item Do things to evaluate FIB.
\item SimBGP.
\item Docker run Quagga
\end{enumerate}
\end{eabstract}

\ekeywords{CABA, A-BGP, Scalable Routing}
