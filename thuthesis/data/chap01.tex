
%%% Local Variables:
%%% mode: latex
%%% TeX-master: t
%%% End:

\chapter{引言}
\label{cha:intro}

\section{研究背景}
现今网络已经成为人们日常生活学习必不可少的一部分,而且规模在不断扩大。未来将出现更多的网络设备和移动终端,网络也将需要更多的IP地址。目前IPv4地址分配基本结束,IPv6 地址逐渐在网络中使用,同时现今网络IPv6向IPv4过渡技术逐渐成熟,都表明IPv6网络将是未来网络的核心,由此产生大规模的网络编址带来的路由的扩展性、稳定性、安全性问题。本文将重点研究路由的可扩展性问题。

随着网络的不断扩大,越来越多的可路由地址块加入了全球BGP的路由表,2015年6月IPv4路由表的规模达到了58 万条\cite{bgptabledata}。 因为IPv6 全球可分配的单播地址(2000::/3)是很大的一块地址空间,况且还有很多未使用的地址空间,IPv6地址前缀的平均长度和最大长度都远远地超过了24,所以IPv6 网络环境下全局路由表的大小在未来可能扩大到无法处理的规模,由此可见解决路由可扩展性的问题迫在眉睫,意义重大。

当前网络路由的可扩展性较差,主要体现两个方面:路由表规模非线性增长、路由结构扁平化造成路由层次间隔离性较差。针对这两个问题,本文对一种层次化IPv6新型编址方案下的基于AS的域间路由算法进行仿真评价和测试,检验该新型编址方案下的新型A-BGP路由策略对网络路由可扩展性的作用。
%不可避免的网络故障会导致网络可达信息和拓扑结构的快速变化,在这种情况下,如果路由不能快速收敛,路由过程中将会发生严重的分组丢失、误发、延迟等问题,严重影响用户体验。

%BGP是目前互联网核心的域间路由协议,其设计中存在缺陷,比较容易被前缀劫持,可能发生路由泄露和恶意攻击。BGP4+是支持IPv6前缀的IPv6网络域间路由协议,因为核心算法没有变化,所以基于BGP4+的IPv6网络域间路由仍旧存在和IPv4相同的问题。

\section{主要工作}
本文对基于AS编址的互联网可扩展路由机制进行了仿真评价与测试,主要进行了以下工作:

\begin{enumerate}
\item 对域间路由协议BGP、BGP-4、BGP4+以及可扩展路由机制的研究现状、现有的IPv4和IPv6网络编址方案进行了综述;
\item 解释基于AS的无类IPv6编址方案和基于自治系统号的BGP协议的实现细节;
\item 为了验证论文中CABA和A-BGP对路由可扩展性的影响,将当前网络环境下的FIB表与A-BGP机制下FIB表进行对比。首先把从Route Views\onlinecite{bgpdata}获取全局RIB表转换成FIB表,然后选取现网络环境下FIB表中的自治系统进行增量部署,将先网络中的FIB表转换成A-BGP下的FIB表,通过对比显示A-BGP下的FIB表项是现网络中FIB 表的10\%;
\item 通过SIMBGP仿真平台,统计全网拓扑结构下自治系统向外宣布前缀的update数目。在A-BGP机制下自治系统只需要向外宣布一条嵌有自治系统号的前缀,而不是现今网络中一个自治系统可能向外宣告多条前缀,最多甚至有4000多条。通过分析显示部署CABA编址方案和使用A-BGP路由策略,增量部署现网络中向外宣布前缀越多的自治系统,向外发布UPDATE包数减少更多;
\item 在Docker软件上部署多台软件路由器,在由Tire1自治系统构成的简单拓扑上实现A-BGP机制,每台软件路由器为一个自治系统,向外宣布一条路由,查看生成的BGP路由表表项数目。然后将其与现互联网中到达Tier1自治系统的FIB表项数目进行对比,发现在CABA编址下A-BGP路由策略中的全局路由表项远少于现网络环境中的全局路由表项。
\end{enumerate}

\section{论文结构}
本文共包含七章:
\begin{itemize}
\item 第一章:引言,介绍研究背景和主要工作;
\item 第二章:相关研究综述,对域间路由协议、可扩展路由机制的研究现状、现有的IP网络编址方案进行了综述;
\item 第三章:基于AS编址的互联网可扩展路由机制框架,描述了基于AS的无类IPv6编址方案和基于自治系统号的BGP 协议;
\item 第四章:基于AS编址的互联网可扩展路由机制的FIB表评估,将当前网络环境下的FIB表与A-BGP机制下FIB表进行对比,对路由的可扩展性进行评估;
\item 第五章:基于AS编址的互联网可扩展路由机制的仿真评价,通过SIMBGP仿真平台,统计全网拓扑结构下自治系统向外宣布前缀的update数目;
\item 第六章:基于AS编址的互联网可扩展路由机制的实现与测试,在Docker软件上部署多台软件路由器,在该拓扑上实现A-BGP机制,查看生成的BGP路由表表项数目;
\item 第七章:总结和进一步研究工作。
\end{itemize}
