
%%% Local Variables:
%%% mode: latex
%%% TeX-master: t
%%% End:

\chapter{引言}
\label{cha:intro}

\section{研究背景}
IPv4地址分配基本结束,IPv6地址逐渐在网络中使用,而且现今网络IPv6向IPv4过渡技术逐渐成熟,都表明IPv6网络将是未来网络的核心,由此产生大规模的网络编址带来的路由的扩展性、稳定性、安全性问题。

随着网络的不断扩大,越来越多的可路由地址块加入了全球BGP的路由表,2015年6月IPv4路由表的规模达到了58 万条\cite{bgptabledata}。因为IPv6 全球可分配的单播地址(2000::/3)是很大的一块地址空间,况且还有很多未使用的地址空间,IPv6地址前缀的平均长度和最大长度都远远地超过了24,所以IPv6 网络环境下全局路由表的大小在未来可能扩大到无法处理的规模,由此可见解决路由可扩展性的问题迫在眉睫、意义重大。

不可避免的网络故障会导致网络可达信息和拓扑结构的快速变化,在这种情况下,如果路由不能快速收敛,路由过程中将会发生严重的分组丢失、误发、延迟等问题,严重影响用户体验。

BGP是目前互联网核心的域间路由协议,其设计中存在缺陷,比较容易被前缀劫持,可能发生路由泄露和恶意攻击。BGP4+是支持IPv6前缀的IPv6网络域间路由协议,因为核心算法没有变化,所以基于BGP4+的IPv6网络域间路由仍旧存在和IPv4相同的问题。

\section{主要工作}
本文对基于AS编址的互联网可扩展路由机制进行了性能评价,主要进行了以下工作:

\begin{enumerate}
\item 对域间路由协议、可扩展路由机制的研究现状、现有的IP网络编址方案进行了综述;
\item 提出了基于AS的无类IPv6编址方案和基于自治系统号的BGP协议;
\item 将当前网络环境下的FIB表与A-BGP机制下FIB表进行对比,对路由的可扩展性进行评估;
\item 通过SIMBGP仿真平台,统计全网拓扑结构下自治系统向外宣布前缀的update数目;
\item 在Docker软件上部署多台软件路由器,在该拓扑上实现A-BGP机制,查看生成的BGP路由表表项数目。
\end{enumerate}

\section{论文结构}
本文共包含七章:
\begin{itemize}
\item 引言,介绍研究背景和主要工作;
\item 相关研究综述,对域间路由协议、可扩展路由机制的研究现状、现有的IP网络编址方案进行了综述;
\item 基于AS编址的互联网可扩展路由机制框架,提出了基于AS的无类IPv6编址方案和基于自治系统号的BGP 协议;
\item 基于AS编址的互联网可扩展路由机制的FIB表评估,将当前网络环境下的FIB表与A-BGP机制下FIB表进行对比,对路由的可扩展性进行评估;
\item 基于AS编址的互联网可扩展路由机制的仿真评价,通过SIMBGP仿真平台,统计全网拓扑结构下自治系统向外宣布前缀的update数目;
\item 基于AS编址的互联网可扩展路由机制的实现与测试,在Docker软件上部署多台软件路由器,在该拓扑上实现A-BGP机制,查看生成的BGP路由表表项数目。
\item 总结。
\end{itemize}
