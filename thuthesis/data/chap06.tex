
%%% Local Variables:
%%% mode: latex
%%% TeX-master: t
%%% End:

\chapter{总结}
\label{conclu}
\section{论文总结}
IaaS云服务在如今愈发重要,大量企业用户由于现有的云服务还不能提供稳定的性能而无法放心将服务器部署在
云上。提供带宽保障是让企业用户将服务器迁移到云上的重要一步,随着这项技术的发展和成熟,必然能够使得
IaaS云服务也有着更大的发展。

现在已经有了大量关于租户带宽保障的研究,然而其中缺少对于租户需求扩展这一问题的深入探讨。然而,
租户需求扩展是在使用云时必然且经常发生的情况,随着租户使用规模的扩大,这一问题发生
会十分频繁。租户需求扩展会带来虚拟机迁移的问题,而对于常见的存储类租户应用,迁移是不能接受的。
之前的租户带宽保障模型均没有能解决这一问题,所以本文提出了一种新的模型,通过巧妙的方法解决了这一问题。

本文的主要工作有:

\begin{enumerate}
\item 对租户带宽保障研究进行了综述,总结了当前已有的带宽保障方案,并介绍了当前主要的几种带宽保障模型。
\item 提出了租户需求扩展对于租户带宽保障问题产生的影响,说明了这一问题的严重性和常见性。
\item 提出了一种新的租户带宽保障模型EBG,通过模型的灵活性,既解决了迁移问题,又使新模型有了更好的性能,
能够更有效的利用云资源。并设计实现了EBG模型的部署算法。
\item 设计实现了一个典型的租户应用,基于云的测试流量生成系统DCTG。介绍了系统的设计目标,给出了详细的设计方案来
满足目标,并实现了系统进行了部署,对系统性能进行了测试说明了系统达成了设计目标。
\item 对EBG模型进行了模拟实验,通过实验说明了EBG模型的优势。同时利用EBG模型对DCTG系统进行了建模,并通过
实验进一步说明了EBG模型的优势。
\end{enumerate}

\section{未来工作展望}
未来的工作可以有几个方面,首先可以基于EBG模型实现一个真实的租户带宽保障方案,这一工作的难点主要在于
结合真实的云数据中心(比如OpenStack)来实现准入控制、虚拟机分配功能,
同时可以结合现有的work-conserving的带宽保障方案实现更加理想的租户带宽保障方案。

其次,可以进一步考察租户需求扩展对于带宽保障的影响,以及研究更好的租户需求扩展和虚拟机迁移算法,
来提高租户网络的性能。
