
%%% Local Variables:
%%% mode: latex
%%% TeX-master: t
%%% End:

\chapter{基于AS新型编址下的FIB表的生成与压缩情况评估}
\label{dctg}
\section{引言}
本章将详细介绍当前网络下FIB表的生成过程以及在新型编址方案CABA下采用基于ASN的BGP协议时FIB表的情况,通过对比评估新型编址环境下路由的可扩展性。
\section{数据集介绍}
实验的数据集来自Oregon大学的Route Views\onlinecite{bgpdata}工程搜集的BGP数据。进入FTP下载页面,选择13台路由器(路由器的名称为: perth, isc, linux, sydney, wide, eqix, saopaulo, nmax, telxati, jinx, sorx,sg),获取其2015 年5 月1 日0 点的全局路由表。

\section{当前网络下的FIB表}

\begin{enumerate}
\item 从Oregon大学的Route Views\onlinecite{bgpdata}下载13台路由器的RIB数据,下载的原始数据是二进制格式,需要工具来解析。
\item 使用libbgpdump\onlinecite{libbgpdump}解析二进制文件。需要从github上下载libbgpdump 的源代码,先执行configure,然后执行make即可生成bgpdump可执行文件,之后通过$./bgpdump -M -O outputfile inputfile$ 命令获取解析后的文件,使用M参数简化路由表项(如:TABLE\_DUMP\_V2|03/31/15 02:00:00|A|206.126.236.120|41095|0.0.0.0/0|41095 3356|IGP)。
\item 相同前缀在RIB表中可能有多条路由信息,但不考虑多路由机制,FIB表中存储的是该前缀的最优路由信息。评价最优路径有两个原则:路径的长度要最短,当路径长度相同的时候,选择下一跳自治系统号最小的路由信息。编写函数,将13台路由器的RIB表转换成FIB表。
\end{enumerate}

\section{基于AS新型编址下FIB表的设计与生成}
\subsection{关键技术}
将现有网络中的FIB表转换成基于AS新型编址CABA下的FIB表的主要过程如下所示:
\begin{enumerate}
\item 1
\item 1
\item 1
\end{enumerate}
\subsection{增量部署}
1
\subsection{遇到的问题和解决方案}
1
\section{FIB表压缩情况的评估}
1
\subsection{压缩结果}
1
\subsection{结果分析}
1
\section{小结}
1


\section{小结}
1
