
%%% Local Variables:
%%% mode: latex
%%% TeX-master: t
%%% End:

\chapter{基于AS新型编址下的FIB表的生成与压缩情况评估}
\label{dctg}
\section{引言}
本章将详细介绍当前网络下FIB表的生成过程以及在新型编址方案CABA下采用基于ASN的BGP协议时FIB表的情况,通过对比评估新型编址环境下路由的可扩展性。
\section{数据集介绍}
实验的数据集来自Oregon大学的Route Views\onlinecite{bgpdata}工程搜集的BGP数据。进入FTP下载页面,选择13台路由器(路由器的名称为: perth, isc, linx, sydney, wide, eqix, saopaulo, nwax, telxatl, jinx, soxrs, sg, kixp),获取其2015年4月3日0点的全局路由表。

\section{当前网络下的FIB表}

\begin{enumerate}
\item 从Oregon大学的Route Views\onlinecite{bgpdata}下载13台路由器的RIB数据,下载的原始数据是二进制格式,需要工具来解析。
\item 使用libbgpdump\onlinecite{libbgpdump}解析二进制文件。需要从github上下载libbgpdump 的源代码,先执行configure,然后执行make即可生成bgpdump可执行文件,之后通过$./bgpdump -M -O outputfile inputfile$ 命令获取解析后的文件,使用M参数简化路由表项(如:TABLE\_DUMP\_V2|03/31/15 02:00:00|A|206.126.236.120|41095|0.0.0.0/0|41095 3356|IGP)。
\item 相同前缀在RIB表中可能有多条路由信息,但不考虑多路由机制,FIB表中存储的是该前缀的最优路由信息。评价最优路径有两个原则:路径的长度要最短,当路径长度相同的时候,选择下一跳自治系统号最小的路由信息。编写函数,将13台路由器的RIB表转换成FIB表。
\end{enumerate}

\section{基于AS新型编址下FIB表的设计与生成}
本小节将详细介绍基于AS新型编址下FIB表设计与生成的关键技术和增量部署的过程。
\subsection{关键技术}
将现有网络中的FIB表转换成基于AS新型编址CABA下的FIB表的主要过程如下所示:
\begin{enumerate}
\item 一个自治系统可能向外宣布多条前缀,在CABA编制下一个自治系统只需要向外宣布一条前缀。我们将当前网络中的FIB表中数据进行分析,每一项数据找到宣布这条前缀的源ASN,如果两个不同IP前缀的源ASN相同,则证明这两个不同IP前缀是由同一个自治系统宣告出来。找出同一个自治系统宣布所有前缀的表项,将其前缀替换成CABA格式的IPv6$/$40前缀。前8位为保留位,选择IANA未使用的地址段10;之后32位为源自治系统的ASN。
\item 因为在现有网络中的FIB表中,有些前缀是由聚合的AS集合宣告的,你并不知道这些前缀究竟是由哪个自治系统向外宣告的。经过对这13台路由器的FIB 表中聚合AS集合数目大于1的表项进行统计,每个路由器约有50万条数据的FIB表中约有50条这样的表项,不足万分之一的数据。为了保证用CABA编址格式下的前缀替换原有前缀的准确性,我们舍弃这不足万分之一的聚合AS集合数目大于1的表项。
\item 替换前缀结束后,在FIB表中同一个自治系统向外公布了一条前缀,但到达这条前缀有多条路径,我们选择最优路径保留下来,其余舍弃。评价最优路径依旧遵循两个原则:路径的长度要最短,当路径长度相同的时候,选择下一跳自治系统号最小的路由信息。编写函数,将13台路由器的RIB表转换成FIB 表。
\end{enumerate}

\subsection{增量部署}
现今互联网架构和规模已经很大,协议和策略已经很完备,并且渗透到经济社会生活的方方面面,将基于AS的新型编址CABA完全部署到互联网上是不现实的,所以需要进行增量部署。本章节对两种增量部署方案的结果进行了评估,两种增量部署的方案如下:
\begin{itemize}
\item 我们得到13台路由器CABA编址下的FIB表,统计每张表的表项,表示该全局路由表中自治系统的个数。

\begin{table}[h]
    \centering
    \caption{路由器中AS的数目}
    \label{tab:routeasnum}
        \begin{tabular}{|c|c|}
            \hline
            路由器名称 & AS数目 \\ \hline
            perth & 913 \\ \hline
            isc   & 50034 \\ \hline
            linx & 49934  \\ \hline
            sydney& 50059  \\ \hline
            wide  & 49882  \\ \hline
            eqix  & 50002  \\ \hline
            saopaulo & 49953 \\ \hline
            nwax  & 49950   \\ \hline
            telxatl  & 49968 \\ \hline
            jinx  & 49685 \\ \hline
            soxrs  & 11880  \\ \hline
            sg    & 49939 \\  \hline
            kixp & 146 \\ 
            \hline
        \end{tabular}
\end{table}

从上表\ref{tab:routeasnum}可以看出perth、soxrs、kixp这三台路由器可能是边界路由器,没有全局路由表的所有信息,所以我在剩余的10台路由器中随机一台路由器的FIB表,从而获取增量部署时所有的ASN,随机路由器的结果为名为saopaulo路由器。
\item 1
\end{itemize}
\section{FIB表压缩情况的评估}
1
\subsection{压缩结果}
1
\subsection{结果分析}
1
\section{小结}
1
